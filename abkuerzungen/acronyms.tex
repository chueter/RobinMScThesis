% Falls das Abkürzungsverzeichnis im Inhaltsverzeichnis angezeigt werden soll
% dann folgende Zeile einkommentieren.
% \addcontentsline{toc}{section}{Abkürzungsverzeichnis}
\section*{\langde{Abkürzungsverzeichnis}\langen{List of abbreviations}}

\begin{acronym}[WYSIWYG]\itemsep0pt %der Parameter in Klammern sollte die längste Abkürzung sein. Damit wird der Abstand zwischen Abkürzung und Übersetzung festgelegt
  \acro{AM}{Activation Maximization}
  \acro{CAM}{Class Activation Mapping}
  \acro{CNN}{Convolutional Neural Network}
  \acro{GD}{Gradient Descent}
  \acro{GTSRB}{German Traffic Sign Recognition Benchmark}
  \acro{KI}{Künstliche Intelligenz}
  \acro{LRF}{Local receptive Fields}
  \acro{MASTIF}{Mapping and Assessing the State of Traffic InFrastructure}
  \acro{NN}{Neural Network}
  \acro{OC}{FOM Online Campus}
  \acro{RGB}{Rot, Grün, Blau}
  \acro{SGD}{Stochastic Gradient Descent}
  \acro{SM}{Saliency Map}
  \acro{StVO}{Straßenverkehrsordnung}
  \acro{WYSIWYG}{What you see is what you get}
\end{acronym}